
\chapter{Experimental Evaluation}
\label{ch:eval}

\section{Counterfactuals}

\subsection{Naive Bayes With Sensitive Attributes}

For the following datapoint:

\resizebox{\textwidth}{!}{
\begin{tabular}{lllllllllll}
\toprule
{} &           age & workclass &  education & marital-status &    occupation & relationship &   race &  gender &          capital-gain & hours-per-week \\
\midrule
46343 &  (31.6, 46.2] &   Private &  Assoc-voc &       Divorced &  Tech-support &    Unmarried &  Black &  Female &  (-4460.355, 16515.0] &   (20.6, 40.2] \\
\bottomrule
\end{tabular}
}

Got the following counterfactuals:

\resizebox{\textwidth}{!}{
\begin{tabular}{lllllllllllrrrr}
\toprule
{} &           age &  workclass &  education &     marital-status &    occupation &   relationship &                race &  gender &        capital-gain & hours-per-week &        O1 &   O2 &  O3 &   O4 \\
\midrule
155 &  (31.6, 46.2] &    Private &  Assoc-voc &  Married-AF-spouse &  Tech-support &      Unmarried &               Black &    Male &  (79128.0, 99999.0] &   (20.6, 40.2] &  0.000000 &  0.7 &   3 &  0.0 \\
162 &  (31.6, 46.2] &    Private &  Assoc-voc &           Divorced &  Adm-clerical &      Unmarried &               Black &    Male &  (79128.0, 99999.0] &   (20.6, 40.2] &  0.000000 &  0.7 &   3 &  0.0 \\
100 &  (31.6, 46.2] &    Private &  Assoc-voc &  Married-AF-spouse &  Adm-clerical &      Unmarried &               Black &    Male &  (58257.0, 79128.0] &   (20.6, 40.2] &  0.000000 &  0.6 &   4 &  0.0 \\
167 &  (31.6, 46.2] &    Private &  Assoc-voc &  Married-AF-spouse &  Adm-clerical &      Unmarried &               Black &    Male &  (79128.0, 99999.0] &   (20.6, 40.2] &  0.000000 &  0.6 &   4 &  0.0 \\
176 &  (31.6, 46.2] &    Private &  Assoc-voc &  Married-AF-spouse &  Adm-clerical &      Unmarried &               Black &    Male &  (58257.0, 79128.0] &   (20.6, 40.2] &  0.000000 &  0.6 &   4 &  0.0 \\
2   &  (60.8, 75.4] &  Local-gov &  Doctorate &           Divorced &             ? &  Not-in-family &  Amer-Indian-Eskimo &  Female &  (79128.0, 99999.0] &   (59.8, 79.4] &  0.000000 &  0.2 &   8 &  0.0 \\
177 &  (31.6, 46.2] &    Private &  Assoc-voc &  Married-AF-spouse &  Tech-support &      Unmarried &               Black &  Female &  (16515.0, 37386.0] &   (20.6, 40.2] &  0.419303 &  0.8 &   2 &  0.0 \\
\bottomrule
\end{tabular}
}


\instructions{
Page budget for Evaluation: 10-15 pages
%
\begin{itemize}
    \item Detail your evaluation methodology, present your results, and provide an analysis of them. Results can be quantitative and/or qualitative (from benchmark, user study, user satisfaction survey, etc.).
    \item It is strongly desired that you have empirical results, nevertheless, this may not be applicable to all types of theses.
\end{itemize}
}

\subsection{Naive Bayes Without Sensitive Attributes}

For the following datapoint:

\resizebox{\textwidth}{!}{
\begin{tabular}{lllllllllll}
\toprule
{} &             age & workclass & education & marital-status & occupation & relationship &   race &  gender &          capital-gain & hours-per-week \\
\midrule
23356 &  (16.927, 31.6] &         ? &   HS-grad &      Separated &          ? &    Unmarried &  Black &  Female &  (-4460.355, 16515.0] &   (20.6, 40.2] \\
\bottomrule
\end{tabular}
} 

We get the following counterfactuals:

\resizebox{\textwidth}{!}{
\begin{tabular}{lllllllllllrrrr}
\toprule
{} &             age &  workclass &  education &     marital-status & occupation &   relationship &                race &  gender &          capital-gain & hours-per-week &        O1 &   O2 &  O3 &   O4 \\
\midrule
128 &  (16.927, 31.6] &  State-gov &    HS-grad &          Separated &          ? &      Unmarried &               Black &    Male &    (79128.0, 99999.0] &   (20.6, 40.2] &  0.000000 &  0.7 &   3 &  0.0 \\
97  &    (31.6, 46.2] &          ? &    HS-grad &          Separated &          ? &  Not-in-family &               Black &    Male &    (79128.0, 99999.0] &   (20.6, 40.2] &  0.000000 &  0.6 &   4 &  0.0 \\
147 &  (16.927, 31.6] &  State-gov &  Doctorate &          Separated &          ? &      Unmarried &               Black &    Male &    (79128.0, 99999.0] &   (20.6, 40.2] &  0.000000 &  0.6 &   4 &  0.0 \\
136 &    (31.6, 46.2] &  State-gov &  Doctorate &  Married-AF-spouse &          ? &        Husband &  Amer-Indian-Eskimo &  Female &  (-4460.355, 16515.0] &   (20.6, 40.2] &  0.165877 &  0.4 &   6 &  0.1 \\
\bottomrule
\end{tabular}
}

\subsection{Fair Bayesian Network}

For the following counterfactual

\resizebox{\textwidth}{!}{
\begin{tabular}{lllllllllll}
\toprule
{} &             age & workclass & education & marital-status & occupation & relationship &   race &  gender &          capital-gain & hours-per-week \\
\midrule
23356 &  (16.927, 31.6] &         ? &   HS-grad &      Separated &          ? &    Unmarried &  Black &  Female &  (-4460.355, 16515.0] &   (20.6, 40.2] \\
\bottomrule
\end{tabular}
}

We get the following counterfactuals

\resizebox{\textwidth}{!}{
\begin{tabular}{lllllllllllrrrr}
\toprule
{} &             age &         workclass &   education & marital-status &       occupation &   relationship &                race &  gender &        capital-gain & hours-per-week &            O1 &   O2 &  O3 &   O4 \\
\midrule
111 &  (16.927, 31.6] &                 ? &        11th &      Separated &                ? &      Unmarried &               Black &  Female &  (79128.0, 99999.0] &   (79.4, 99.0] &  0.000000e+00 &  0.7 &   3 &  0.0 \\
114 &  (16.927, 31.6] &                 ? &     HS-grad &      Separated &                ? &      Unmarried &               White &  Female &  (79128.0, 99999.0] &   (79.4, 99.0] &  0.000000e+00 &  0.7 &   3 &  0.0 \\
162 &  (16.927, 31.6] &                 ? &        11th &      Separated &                ? &      Unmarried &               White &  Female &  (79128.0, 99999.0] &   (20.6, 40.2] &  0.000000e+00 &  0.7 &   3 &  0.1 \\
58  &  (16.927, 31.6] &      Never-worked &     HS-grad &      Separated &                ? &      Unmarried &  Asian-Pac-Islander &  Female &  (58257.0, 79128.0] &   (79.4, 99.0] &  0.000000e+00 &  0.6 &   4 &  0.0 \\
118 &  (16.927, 31.6] &                 ? &  Assoc-acdm &      Separated &                ? &      Unmarried &               Black &    Male &  (79128.0, 99999.0] &   (79.4, 99.0] &  0.000000e+00 &  0.6 &   4 &  0.0 \\
172 &  (16.927, 31.6] &      Never-worked &        11th &      Separated &                ? &      Unmarried &               White &  Female &  (58257.0, 79128.0] &   (20.6, 40.2] &  0.000000e+00 &  0.6 &   4 &  0.1 \\
\end{tabular}
}

\subsection{Fair Tree Classifier}


\section{Experimental Setup}
\label{sec:eval:expsetup}

\instructions{
\begin{itemize}
    \item Explain the methodology used for evaluating your contribution, and the metrics used for evaluation.
    \item If you use any dataset, explain it, detail its version, and mention briefly some main statistics about it, of interest for your problem (e.g., size, provenience, etc.), if appropriate.
    \item If you collect ground truth data, describe your annotation experiment. Explain what the annotators were asked to do (and show a screenshot or schema if available). Detail the number of annotators, their nature (experts, or crowdworkers), the criteria for deciding on each annotation instance (e.g., majority class, dynamic judgments, etc.), the criteria for ensuring quality (e.g., minimum accuracy, filters). If possible, report the inter-annotator agreement coefficient and mention how strong this value means that the agreement is.
\end{itemize}
}

\section{Experimental Results}
\label{sec:eval:results}


\instructions{
\begin{itemize}
    \item Present the results, using tables and (pretty) plots.
\end{itemize}
}

\section{Analysis}
\label{sec:eval:analysis}

\instructions{
\begin{itemize}
    \item Now that you presented the results, what do these results actually mean (esp. regarding the objectives you set out in the introduction)? 
    \item Can you identify success and failure cases? 
    \item What do the results say for individual parts you evaluate and overall in combination? 
    \item Make sure you formulate clear take-home messages.
\end{itemize}
}
